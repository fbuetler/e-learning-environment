\chapter{Introduction}

\section{Motivation and Background}

With the introduction of Lehrplan 21 Computer Science became an integral part of the Swiss education curriculum \cite{Lehrplan21}. Pupils learn to understand the basic concepts of Computer Science and how to use them for problem solving. These concepts include methods on how to process, evaluate and summarize data, how to securely communicate and how to develop solution strategies for simple problems of information processing \cite{MedienUndInformatik}. The Education and Counselling Center for Computer Science Education at ETH Zürich \cite{ABZ} supports schools to teach these concepts among others by providing teaching resources and learning environments.

\section{Goals}

The main goal in this bachelor thesis is to implement tasks and riddles based on the textbook “einfach Informatik 3/4”, that will be published in spring 2021 \cite{EinfachInformatik}, in a computer-based learning environment that teaches the following concepts:

\begin{itemize}
    \item representing information with symbols,
    \item keeping information secret and
    \item learning from data
\end{itemize}

for pupils in the second cycle i.e the third and fourth grade of elementary school.
Along with solving tasks and riddles about the mentioned topics the ability of reading, writing, counting and calculating is trained as well.

\section{Related Work}

\TODO{maybe: about other learning environments}

\section{Outline}

This report start with the architecture of the computer based learning environment and its core implementation. Next it dedicates a chapter to each of the aforementioned concepts. Each chapter explains first how the concept is thought by hands-on exercises, then it gives an in-depth technical insight on how these exercises are implemented. At this point it is needed to be mentioned that the shown code might be simplified to highlight the intersting aspects, improve readability and to keep this report consise. Testing and Continous integration of the project is discussed before, finally, the report ends with a conclusion of the project.
