\chapter{Introduction}

\section{Motivation and Background}

With the introduction of Lehrplan 21 Computer Science became an integral part of the Swiss education curriculum \cite{Lehrplan21}. Pupils learn to understand the basic concepts of Computer Science and how to use them for problem solving. These concepts include methods on how to process, evaluate and summarize data, how to securely communicate and how to develop solution strategies for simple problems of information processing \cite{MedienUndInformatik}. The Education and Counselling Center for Computer Science Education at ETH Zurich (ABZ) supports schools to teach these concepts among others by providing teaching resources and learning environments.

\section{Goals}

The main goal in this bachelor thesis is to implement tasks and riddles based on the textbook “einfach Informatik 3/4” in a computer-based learning environment that teaches the following concepts:
\begin{itemize}
    \item representing information with symbols,
    \item keeping information secret and
    \item learning from data
\end{itemize}
for pupils in the second cycle.
Along with solving tasks and riddles about the mentioned topics the ability of reading, writing, counting and calculating is trained as well.

\section{Related Work}

\TODO{abz.inf.ethz.ch other learning environments}

\section{Outline}

This report first explains how the aforementioned concepts are thought by hands-on exercises, then gives in-depth technical insight on how a learning environment is developed and how these exercises are implemented. Finally, the report ends with a conclusion with a review of the project.
