\chapter{Introduction}

\section{Motivation and Background}

With the introduction of Lehrplan 21, Computer Science became an integral part of the Swiss education curriculum \cite{Lehrplan21}. Pupils learn to understand the basic concepts of Computer Science and how to use them for problem-solving. These concepts include methods on how to process, evaluate and summarize data, how to securely communicate and how to develop solution strategies for simple problems of information processing \cite{MedienUndInformatik}. The Education and Counselling Center for Computer Science Education at ETH Zürich \cite{ABZ} supports schools to teach these concepts among others by providing teaching materials and learning environments.

\section{Goals}

The main goal in this bachelor thesis is to implement tasks and riddles based on the textbook “einfach Informatik 3/4”, that will be published in spring 2021 \cite{EinfachInformatik}, in a computer-based learning environment that teaches the following concepts:

\begin{itemize}
    \item Representing information with symbols,
    \item Keeping information secret and
    \item Learning from data
\end{itemize}

for pupils in the second cycle, i.e the third and fourth grade of elementary school in German.
Along with solving tasks and riddles about the mentioned topics, reading, writing, counting and calculating skills are trained, too.

\section{Related Work}

\TODO{maybe: about other learning environments}

\section{Outline}

This report starts with the architecture of the computer-based learning environment and its core implementation. Next, it dedicates a chapter to each of the mentioned concepts. Each chapter first explains how the concept is taught by hands-on exercises, then it gives an in-depth technical insight on how these exercises are implemented. At this point it needs to be mentioned that the shown code might be simplified to highlight the interesting aspects, improve readability and to keep this report concise. Testing and Continuous Integration of the project is discussed before the report is rounded off with a conclusion of the project.

The source code, tests, this report and all other used scripts and documents can be found on the ETH GitLab at \url{https://gitlab.ethz.ch/flbuetle/bsc-thesis}.