\chapter{Conclusion}
\label{chapter:conclusion}

The main goal in this bachelor thesis was to implement tasks and riddles based on the textbook “einfach Informatik 3/4” in a computer-based learning environment for pupils in the second cycle. The concepts covered in this thesis are:

\begin{itemize}
    \item representing information with symbols,
    \item keeping information secret and
    \item learning from data
\end{itemize}

\section{Obstacles}
\label{section:obstacles}

\TODO{framework: how to structure the code}
\TODO{word generation}

\section{Limitation}
\label{section:limitation}

An interesting aspect of this thesis was to implement so many different exercises teaching different concepts. However, since not all exercises of each topic in the textbook “einfach Informatik 3/4” were implemented, I consider this as a major drawback. Some exercises like \nameref{section:similarWords} are preparatory exercises for more advanced exercises like communication with damaged messages. For the next time I thing it would improve the overall learning experience of pupils if a whole topic is covered and not only a fraction of it.

\TODO{maybe also: own learning environment}

\section{Future Work}
\label{section:futureWork}

The future work is directly derivated from its limitations.

A big imporovement would be to cover the remaining fraction of the exercises that were not covered in this learning environment. Especially, those that are build on exercises already implemented in this learning environment.

\TODO{maybe: integrate into existing learning environment}