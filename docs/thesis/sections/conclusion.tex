\chapter{Conclusion}
\label{chapter:conclusion}

The main goal in this bachelor thesis was to implement tasks and riddles based on the textbook “einfach Informatik 3/4” in a computer-based learning environment for pupils in the second cycle. The concepts covered in this thesis are:

\begin{itemize}
    \item representing information with symbols,
    \item keeping information secret and
    \item learning from data
\end{itemize}

\section{Obstacles}
\label{section:obstacles}

One of the first obstacles I encountered, was how to structure the code. Vue.js is much less opionated than for example Angular and therefore many possible ways exist on how to structure an application. On one hand, one can enjoy the freedom that vue.js gives and most of the time any approach will work on to structure the code. On the other hand, this may backfire later down the road, when the application grew and no real struture is defined. One think I really missed in this project are code reviews by other people, knowing vue.js, to give me feedback about the structure and the logic. Because code reviews by experienced people can pin point one in the right direction and avoid having nightmares later on.

One thing that at first glance looked like a straight forward problem and later on turn out to be a hard one, was the generation of allowed words in the \ref{section:similarWords} exercises. As already showed, many approaches were taken until a satisfying result was achieved. The problem here was that the application needs to know all possible words pupils can think of, since having an exercise evaluated as incorrect even though the word exists is just demotivating. In the end, having a list of about one millions allowed words will hopeful cover this.

\section{Limitation}
\label{section:limitation}

An interesting aspect of this thesis was to implement so many different exercises teaching different concepts. However, since not all exercises of each topic in the textbook “einfach Informatik 3/4” were implemented, I consider this as a major drawback. Some exercises like \nameref{section:similarWords} are preparatory exercises for more advanced exercises like communication with damaged messages. For the next time I thing it would improve the overall learning experience of pupils if a whole topic is covered and not only a fraction of it.

\TODO{maybe also: own learning environment}

\section{Future Work}
\label{section:futureWork}

The future work is directly derivated from its limitations.

A big imporovement would be to cover the remaining fraction of the exercises that were not covered in this learning environment. Especially, those that are build on exercises already implemented in this learning environment.

\TODO{maybe: integrate into existing learning environment}