\chapter{Evaluation of Different Frameworks}

\TODO{intro to frontend frameworks}
% history: javascript more and more used in webpage to create dynamic pages, code moved from server to browser, no formal organization of html, css and js files, therefore use frontend frameworks

Frontend frameworks allow to easily create dynamic websites. Today, a wide range of frontend frameworks are available and hence it is not easy to choose one. Different factors come into play when selecting a frontend framework, such as learning curve, user experience, community activty and many more. In this section the most popular frameworks are introduced and the reason why Vue.js was chosen.

\section{Angular}

Angular is an open source frontend framework developed by Google, its community and other companies and is based on Typescript. It was published in 2016 and is the successor of Angular.js. Angular allows to create progressive web apps, with high performance in a short time \cite{Angular}.

\TODO{pros and cons per framework}
% long term support guaranteed by google
% big community
% ability to create complex apps that scale well
% steep learning curve
% https://www.simform.com/best-frontend-frameworks/
% https://www.keycdn.com/blog/frontend-frameworks 

\section{React}

Maintained by Facebook, React is an open source front end framework for building user interfaces. Its initial release was published in 2013 and works with components. A component encapsulates its logic from other components and can manage its own state. Multiple components can be composed to easily create a large, complex application \cite{React}.

\section{Vue.js}

Vue.js is developed by a company, but by Evan You and was first published in 2014. It is an alternative to Angular and React and was supossed to be a lightweight version of Angular. Vue.js is also based on reusable components, each having its own HTML, Javascript and CSS \cite{Vue}.

% 2020 developer survey
% 35% use react, 25% use angular, 17% use vue.js
% developers that used a framework and are going to contiouing using it
% 69% react, 54% angular, 66% vue.js
% developers that used a framework and dont want to use it anymore
% react 31%, angular 54%, vue.js 34%, 
% github stars
% react 164k, angular 71k, vue.js 179k

\section{Conclusion}

All three frontend frameworks are capable of creating the computer based learning environment that is intended in the scope of this thesis. Overall, it is a question of preference what frontend framework one wants to use. At the point the development of the learning environment started, Vue.js version 3 was already published. However, Vue.js version 2 is used in this thesis, because the ecosystem has no caught up yet and many libraries only work with version 2 at the moment.

% typescript
% jest
% nightwatch
% CI/CD
% 
% basic functionality: click, drag&drop, undo, trashcan, difficulty, check, next
% tutorial with video
% styling like highlighting, locked
% game component and interface for games
% sudoku solver
% words generating
% drawing on canvas
% number systems
